% !TEX root = ./main.tex
\section{Introduction}
\label{sec:intro}

Work stealing (WS) is a distributed on-line scheduling algorithm
proposed for shared-memory multi-cores~\cite{Arora2001}. With work
stealing, each processor maintains a local list of tasks to be
executed. When a processor has executed all its local work, it probes
others by sending work requests to randomly chosen processors. We say
that this processor attempts to ``\emph{steal}'' work from others.
Despite its apparent simplicity, work stealing has been proven to be
very effective for scheduling tasks in shared-memory homogeneous
architectures~\cite{Blumofe1999}. Today, the research on WS is driven
by the question on how to extend the analysis for the characteristics
of new computing platforms (distributed memory, large scale,
heterogeneity).  Notice that beside its theoretical interest, WS has
been implemented successfully in several languages and parallel
libraries including Cilk~\cite{Leiserson1998,Leiserson2009}, TBB
(Threading Building Blocks)~\cite{Robison2008}, the PGAS
language~\cite{Dinan2009,Seung-Jai2011} and the KAAPI run-time
system~\cite{Kaapi2007}.

% The performance of a scheduling algorithm is defined by the
% \textit{Makespan}, defined as the maximum completion time of the
% parallel application.

In this paper, we study the performance of WS in a distributed-memory
context.
Such an architecture consists of independent processing elements with
private local memories linked by an interconnection
network. Communication latency is crucial and highly influences the
performance of the applications~\cite{MCMCA}. 
% However, there are only few works dealing with optimized allocation strategies and the relationships with the allocation and scheduling process is most often ignored.
The impact of scheduling is important since the whole execution can be
highly affected by a large communication latency of interconnection
networks~\cite{MultiCoreClusterArchitectur2015}.


%Principle of WS
% WS is an efficient scheduling mechanism targeting medium range parallelism of multi-cores for fine-grain tasks.
% Its principle is briefly recalled as follows:
% each processor manages its own (local) list of tasks.
% When a processor becomes idle, it randomly chooses another processor and steals some work (if possible).
% Its analysis is probabilistic since the algorithm itself is randomized.





\paragraph{Contributions}%  We present a theoretical analysis for an
% upper bound of the expected makespan and we run a complementary series
% of simulations in order to assess how this new bound behaves in
% practice depending on the value of the latency.
In this work, we study how communication latency impacts work
stealing.  Our work has four main contributions. First, we create a
new realistic scheduling model for distributed-memory clusters of $p$
identical processors including latency (denoted by $\lambda$).  Second,
we provide an upper bound of the expected makespan to execute a bag of
$\calW$ independent unitary tasks by $p$ processors.  This bound is
the sum of two terms. The first is the usual lower bound on the best
possible load-balancing~$\frac{\mathcal{W}}{p}$; The second term is
proportional to $\lambda\log(\frac{\mathcal{W}}{\lambda})$.
%The additional term depends on the dependence model. In the case of independent tasks, this additional term is
%$16\lambda\log_2(\frac{\mathcal{W}}{\lambda})$.
%In the case of DAG, this term is proportional to the critical path.
Third, we develop a lightweight Python simulator for running adequate
experiments. Our simulator is developed to be flexible enough to
simulate different topologies and applications with different variants
of Work Stealing algorithms.
%This simulator uses the Work Stealing algorithm to schedule an amount of
%work W in a distributed platform composed of p
%identical processors. 
The code is available on GitHub (\url{https://wssimulator.github.io})
along logs of presented experiments. The simulator is
generic enough to be used in different contexts of online
scheduling and interfaces with standard trace analysis
tools
Using this simulator,
we provide simulation results that challenges this bound.
%in the case of indepedent tasks.
These experiments show that the second additional term has indeed the
form $c\lambda\log_2(\frac{\mathcal{W}}{\lambda})$, with
$c\approx3.8$. Note that our theoretical upper bound shows that
$c\le16.12$.  Finally, we show that our methodology can be easily
extended to the case of tasks with precedence. We provide an upper
bound of the expected makespan in this case. This bound is also
composed of two terms. The first is the usual lower bound
$\frac{\mathcal{W}}{p}$ and the second additional term is proportional to the
critical path ($6\lambda D$)

Our analyses are based on the use of adequate potential
functions. While the overall approach is similar to existing work, and in
particular \cite{Denis2013}, there are two technical difficulties that
had to be lifted.  First, a natural extension of the potential
functions used in \cite{Denis2013} does not work as one needs to
account for the work in transit. Second, we do not assume that task
execution and steal requests are synchronized.  This leads to consider
time steps of duration equal to the communication latency.

The rest of the paper is organized as follows.  We review the related works
in Section~\ref{sec:relatedworks}. We present the independent task
model in Section~\ref{sec:wsmechanisms}. We derive our main technical
result in Section~\ref{sec:analysis_indep} in which we obtain a bound
on the makespan. We introduce our work stealing simulator in
Section~\ref{sec:experiemnts} that we use to study the limits of our
analysis. We show how to extend our analysis to tasks with precedence
in Section~\ref{sec:dagOfPrecedence} and conclude in
Section~\ref{sec:conclusion}.  Last, Appendix~\ref{sec:proofs} contains
some technical proofs. 

\section{Related Work}
\label{sec:relatedworks}

Let us first introduce the problem of scheduling:
It is the process which aims at determining where and when to execute the tasks of a target parallel application.
The applications are represented as directed acyclic graphs where the vertices are the basic operations and the arcs are the dependencies between the tasks~\cite{CosnardTrystram92}.
Scheduling is a crucial problem which has been extensively studied under many variants for the successive generations of parallel and distributed systems.
The most commonly studied objective is to minimize the makespan (denoted by $C_{\max}$) and the underlying context is usually to consider centralized algorithms.
This assumption is not always realistic, especially if we consider distributed memory allocations and an on-line setting.
% (called \emph{Makespan} and denoted by $C_{\max}$)


Work Stealing (WS) is a distributed version of scheduling that received sustained attention over the last twenty years. 
It has been studied originally by Blumofe and Leiserson
in~\cite{Blumofe1999}.  They showed that the expected Makespan of a
series-parallel precedence graph with $\mathcal{W}$ unit tasks on $p$
processors is bounded by
$E(C_{\max}) \leq \frac{\mathcal{W}}{p}+\mathcal{O}(D)$ where $D$ is
the length of the critical path of the graph (its depth).  This
analysis has been improved in Arora \textit{et al.}~\cite{Arora2001}
using potential functions.  The case of varying processor speeds has
been studied by Bender and Rabin in~\cite{Bender2002} where the
authors introduced a new policy called \textit{high utilization
  scheduler} that extends the homogeneous case.  The specific case of
tree-shaped computations with a more accurate model has been studied
in~\cite{Sanders1999}.  However, in all these previous analyses, the
precedence graph is constrained to have only one source and an
out-degree of at most 2 which does not easily model the basic case of
independent tasks.  Simulating independent tasks with a binary
precedences tree gives a bound of
$\frac{\mathcal{W}}{p}+\mathcal{O}(\log_2 (\mathcal{W}))$ since a
complete binary tree of $\mathcal{W}$ vertices has a depth
$D \leq \log_2 (\mathcal{W})$.  However, with this approach, the
structure of the binary tree dictates which tasks are stolen. More
recently, in \cite{Denis2013}, Tchiboukjian \textit{et al.} provided
the best bound known at this time:
$\frac{\mathcal{W}}{p}+c.(\log_2 \mathcal{W})+\Theta(1)$ where
$c\approx3.24$.

Acar \emph{et al.}~\cite{Acar2000,Acar2002} studied data locality of
WS on shared-memory and focus on cache misses.  The underlying model
assumes that the execution time takes $m$ time units if the
instruction incurs a cache miss and $1$ unit otherwise.  A steal
attempt takes at least $s$ and at most $ks$ steps to complete ($k\ge1$
multiple steal attempts before succeeding).  When a steal succeeds,
the thief starts working on the stolen task at the next step.  This
model is similar to the classical WS model without communication since
the thief does not wait several time steps to receive the stolen task.

In complement,~\cite{Gast2010,Mitzenmache1998} provided a theoretical
analysis based on a Markovian model using mean field theory. They
targeted the expectation of the average response time and showed that
the system converges to a deterministic Ordinary Differential
Equation.  Note that there exist other results that study the steady
state performance of WS when the work generation is random including
Berenbrink \textit{et al.}~\cite{Berenbrink2003}, Lueling and
Monien~\cite{Lueling1993} and Rudolph \textit{et al.}~\cite{Rudolph1991}.  
%\bigskip

%extension vers les comm


In all these previous theoretical results, communications are not directly addressed
(or at least they are taken implicitly into account by the underlying model).
Most studies on WS largely focused on shared memory systems and 
its performance on modern platforms (distributed-memory systems, hierarchical platform, clusters with explicit communication cost) is not really well understood. 
The difficulty lies in the problems of communication which become more crucial in modern platforms~\cite{Arafat2016}.
Let us briefly review the most relevant works dealing with WS with communications.

%Furthermore, the communication issues become even more crucial in heterogeneous distributed systems
%In shared-memory systems the thieves constantly attempt to asynchronously steal work from randomly chosen victims
%until they find work. In distributed-memory systems, it is much harder since the thieves cannot autonomously steal
%work from a victim without disrupting its execution~\cite{Saraswat2011}.
%WS algorithms are based on shared memory multicore chips, 
Dinan \textit{et al.}~\cite{Dinan2009} implemented WS on large-scale clusters,
and proposed to split each tasks queues into a part accessed asynchronously by local processes 
and a shared portion synchronized by a lock which can be access by any remote processes in order to reduce contention.
%In-place task creation which optimize critical operations. 
%Multi-core processor based on NUMA (non-uniform memory access) architecture is the mainstream today and such new platforms include accelerators as GPGPU~\cite{Yang2017}. 
The authors propose an efficient task management mechanism which is to divide the application into a large
number of fine grained tasks which generate a large amount of small communications between the CPU and the GPGPU accelerators.
However, these data transmissions are very slow.
They consider that the transmission time of small data and large data is the same. 


Besides the large literature on theoretical works, there exist more practical studies implementing WS libraries where some attempts were provided for taking into account communications:
%practical
SLAW is a task-based library introduced in~\cite{Gue2010}, combining work-first
and help-first scheduling policies focused on locality awareness 
in PGAS (Partitioned Global Address Space) languages like UPC (Unified Parallel C). 
It has been extended in HotSLAW, which provides a high level API that abstracts concurrent task management~\cite{Seung-Jai2011}.
\cite{Shigang2013} proposes an asynchronous WS (AsynchWS) strategy which exploits opportunities to overlap communication with
local tasks allowing to hide high communication overheads in distributed memory systems. 
The principle is based on a hierarchical victim selection, also based on PGAS.
Perarnau and Sato presented in~\cite{swann2014} an experimental evaluation of WS on the scale of ten thousands compute nodes where the communication depends on the distance between the nodes.
They investigated in detail the impact of the communication on the performance. In particular, the physical distance between remote nodes is taken into account.
Mullet \textit{et al.} studied in~\cite{Muller2106} 
Latency-Hiding, a new WS algorithm that hides the overhead
caused by some operations, such as waiting for a request from a client or waiting
for a response from a remote machine.
The authors refer to this delay as \emph{latency} which is slightly
different that the more general concept we consider in our paper. 
Agrawal \textit{et al.} proposed in~\cite{Agrawal} an analysis showing the optimality for task graphs with bounded degrees
and they developed a library in Cilk++ called \textit{Nabbit} for executing tasks
with arbitrary dependencies, with reasonable block sizes.

%\bigskip

%\cite{Jixiang Yang xx }

%Traditional work stealing strategy mainly verifies its performance by simulation data
%or individual applications and concludes corresponding results. Parallel
%computing becomes popular in multi-core era, and it is not persuasive to verify
%scheduling performance just by selecting experimental data for individual applications,
%but needs to have a lot of applications to test its performance.

%Med : in our paper we study the work stealing in general, so the theoretical results covers several a lot of applications that use work stealing with communication cost  



%%% Local Variables:
%%% mode: latex
%%% TeX-master: "main"
%%% End:
