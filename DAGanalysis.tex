% !TEX root = ./main.tex
\section{Extension to a Model of Tasks with Precedence}
\label{sec:dagOfPrecedence}
%\subsubsection{DAG of precedence}

In this section, we show that the analysis presented in
Section~\ref{sec:analysis_indep} can be adapted to the case of tasks
with precedence constraints. The main difficulty here is to
redefine a new potential function, which we do by combining the ideas
of \cite{Arora2001,Denis2013} to our model of latency of
Section~\ref{sec:analysis_indep}. This demonstrates the power of an
approach by potential function. 

\subsection{Model}

We consider a model similar to \cite{Arora2001,Denis2013} where the
workload is composed of $\mathcal{W}$ unitary tasks that have
precedence constraints represented by a DAG (Directed Acyclic Graph). 
This DAG has a single source that represents the first task and that
is originally located on a given processor.
We consider that the scheduling is done as in
\cite{Arora2001}: each processor maintains a double-ended queue
(called deque) of activated tasks. If a processor has one or more task
in its deque, it executes the task at the head of its deque. This
takes one unit of time. After completion, a task might activates $0$,
$1$ or $2$ tasks that are pushed at the end of the deque.  The
activation tree is a binary tree whose shape depends on the execution
of the algorithm. It is a subset of the original DAG and has the same
critical path.  We define the height of node of this tree as
follows. The height of the source as $D$ (\emph{i.e.}, the length of
the critical path). The height of another task is equal to the length
of its father minus one.  We assume that when a processor steals work
from another processor, it steals the activated tasks with the largest
height.  In the case of the DAG of precedence, a processor can only
answer positively if it has two or more activated tasks in its deque.
In this case, it sends its task that has the largest height.

\subsection{Potential function}
The potential will depend on the maximal height of
the tasks that a processor has in its lists. Denoting by $h_i(k)$ the
maximal height of the tasks that processor $i$ has in its deque, we
define a quantity $w_i(k)$, that we call the potential work of
processor $i$, as:
\begin{align*}
  w_i(k)=\left\{
  \begin{array}{ll}
    (2\sqrt{2})^{h_i(k)} & \text{if the processor
                           $i$ has two or more tasks in its deque,}\\
    \frac12(2\sqrt{2})^{h_i(k)}& \text{if the processor
                                 $i$ has only one task in its deque,}\\
    0 & \text{if the processor
        $i$ does not have any tasks.}
  \end{array}
        \right.
\end{align*}
For the tasks in transit, we define the potential work in transit
$s_i(k)=\frac12(2\sqrt{2})^{h}$, where $h$ is the height of the task
in transit.

The potential is then defined similarly as in
Equation~\eqref{eq:potential_indep}:
\begin{align}
  \label{eq:potential_DAG}
  \phi(k) = 1+ \sum_i \phi_i(k)\qquad\text{where }\phi_i(k)=w^2_i(k) + 2s^2_i(k).
\end{align}
We are now ready to prove the following lemma, that is an analogue of
Lemma~\ref{lem:indep} for the case of dependent tasks. 
\begin{lemma}\label{lem:DAG}
  For the case of DAG, the expected ratio between
  $\phi(k+1)$ and $\phi(k)$ knowing $\mathcal{F}_{k}$ is bounded by:
  \begin{equation*}
    \mathbb{E}[\phi(k+1) \:|\: \mathcal{F}_{k}] \leq
    h(r(k))\phi(k),
  \end{equation*}
  where the potential is defined as in
  Equation~\eqref{eq:potential_DAG} and $h(r)$ is as in
  Lemma~\ref{lem:indep}, \emph{i.e.}, 
  $h(r)=\frac34 + \frac14 \left(\frac{p-2}{p-1} \right)^{r}$.
% Lemma~\ref{lem:indep} also holds when the tasks have a DAG of
% precedence, with a potential function $\phi$ defined as in
% Equation~\eqref{eq:potential_DAG}. 
\end{lemma}
\begin{pf}
  Similarly to the proof of Lemma~\ref{lem:indep}, we study the
  expected decrease of the potential by distinguishing three cases:
  the case $s_i>0$ (work arriving at a thief) and the cases where a
  processor is or becomes idle correspond to cases 1 and 4 of
  Lemma~\ref{lem:indep} and can be analyzed exactly as what was done
  since Cases~1 and 4 of the proof of Lemma~\ref{lem:indep} do not
  depend on the task dependence model.  Moreover, the distinction
  between Case~2 and Case~3 that was done for independent tasks is not
  necessary for DAGs as there is no steal threshold for DAG. Last, the
  analysis of Case~2b of Lemma~\ref{lem:indep} is similar: if a
  processor does not receive any work request, the potential does not
  grow.

  Hence, in the reminder of the proof, we focus on the only
  interesting case which what happens when a processor receives one or
  more work request (Case~2b of the proof of Lemma~\ref{lem:indep}).
  We distinguish two cases depending on how many tasks are in the
  deque of processor $i$ when it receives a work request.
  \begin{enumerate}
  \item If processor $i$ has only one task (say of height $h$), then
    it cannot send work. In this case, it will complete
    its tasks that will activate at most $2$ tasks of height $h-1$. In
    such a case, the potential work of processor $i$ will go from
    $w_i(k)=\frac12(2\sqrt{2})^h$ to at most
    $w_i(k+1)=(2\sqrt{2})^{h-1}$. This shows that 
    \begin{align*}
      \phi_i(k+1) = w^2_i(k+1)\le (2\sqrt{2})^{2h-2} = (2\sqrt{2})^{2h}/8 
      = w_i(k)^2/2 \le \frac12 \phi_i(k). 
    \end{align*}
  \item If processor $i$ has two or more tasks, then by construction,
    it can have at most two tasks of maximal height (say $h$). In this
    case, the processor will send one task (of height $h$) to the
    thief, and will at the mean time execute the other task. In this
    case, the maximal height of the task of its deque will be $h-1$
    and the task sent will be of height $h$. This implies that
    $w_i(k+1)\le (2\sqrt{2})^{2h-1}\le w_i(k)/(2\sqrt2)$ and
    $s_i(k+1)\le \frac12(2\sqrt{2})^{h}\le w_i(k)/2$. This shows
    that:
    \begin{align*}
      \phi_i(k+1) = w^2_i(k+1)+2s^2_i(k+1) \le w^2_i(k)/8 + 2w^2_i(k)/4 
      = \frac58 \phi_i(k). 
    \end{align*}
  \end{enumerate}
  As both $1/2$ and $5/8$ are strictly smaller than the $3/4$ of
  Equation~\eqref{eq:proof_indep_34}, the rest of the proof of
  Lemma~\ref{lem:indep} can be applied \emph{mutatis mutandis} to
  prove Lemma~\ref{lem:DAG}.
\end{pf}

\subsection{Analysis of the Makespan}
\label{subsec:analysisMakspan}


% \subsection{Bound on the Number of Work Requests}
% \label{def:gamma}

The bound on the expected decrease of the potential of
Lemma~\ref{lem:DAG} is the same  as the bound of Lemma~\ref{lem:indep}
for independent tasks. Since the proof of Lemma~\ref{lem:OfR} does not depend on the dependence
structure of the tasks but only the bound on the expected decrease of
Lemma~\ref{lem:indep}, this implies that Lemma~\ref{lem:OfR} also
applies for the case of tasks with precedence.

We are now ready to prove the bound on the total completion time
$C_{\max}$ for the DAG model that is summarized by the following theorem:
\begin{theorem}
  \label{theo:dag_cmax}
  Let $C_{max}$ be the Makespan of $\mathcal{W}$ unit tasks with a DAG
  of precedence that has a critical path of $D$ scheduled by WS with
  latency $\lambda$.  Then
  \begin{align*}
    (i)  \qquad &\esp{C_{max}} \leq \frac{\mathcal{W}}{p} +
                  6\lambda\gamma D; \\
    (ii) \qquad  &\Proba{ C_{max}  \ge \frac{\mathcal{W}}{p} +
                   6\lambda\gamma D  + x} \le
                   {2}^{-x/(2\lambda\gamma)}.
  \end{align*}
  The constant $\gamma$ is the same as in Lemma~\ref{lem:OfR}. In
  particular $\gamma<4.03$.
\end{theorem}
\begin{pf}
  The case of DAG is similar as the independent tasks model %(proof~\ref{proof_cmax}),
  the only difference being the expression of the potential at time $0$.
  In the case of a DAG, the potential at time $0$ is equal to
  $1+\frac14(2\sqrt{2})^{2D}=1+\frac148^D\le 8^D$ where $D$ is the
  critical path.  This shows that $\log_2\phi(0)\le D\log_28=3D$ which
  implies that
  \begin{align*}
    \esp{\mathcal{C}_{max}}
    & \leq \frac{\calW}{p} + 6\lambda\gamma D.
  \end{align*}
\end{pf}
The bounds that we obtained in Theorem~\ref{theo:dag_cmax} and Theorem~\ref{theo:cmax} are composed of
a term due to the computation of tasks ($\calW/p$) and an overhead
term related to the depth.  In case of independent tasks the depth is
$\log(W)$ but we obtain a slightly better bound due to tasks not being
divided below $\lambda$.  We actually believe that the bounds on the DAG could be
refined when considering non-unit tasks.


%%% Local Variables:
%%% mode: latex
%%% TeX-master: "main"
%%% End:
